\documentclass[12pt,a4paper]{article}
\usepackage[utf8]{inputenc}
\usepackage[czech]{babel}
\usepackage[T1]{fontenc}
\usepackage{amsmath}
\usepackage{amsfonts}
\usepackage{amssymb}
\usepackage{graphicx}
\usepackage{titlesec}
\usepackage[left=2cm,right=2cm,top=2cm,bottom=2cm]{geometry}
\usepackage{indentfirst}
\usepackage{listings}
\usepackage{color}
\usepackage{array}

%Pravidlo pro řádkování
\renewcommand{\baselinestretch}{1.5}

%Pravidlo pro začínání kapitol na novém řádku
\let\oldsection\section
\renewcommand\section{\clearpage\oldsection}

%Formáty písem pro nadpisy (-změněno na bezpatkové \sffamily z původního \normalfont
\titleformat{\section}
{\sffamily\Large\bfseries}{\thesection}{1em}{}
\titleformat{\subsection}
{\sffamily\large\bfseries}{\thesubsection}{1em}{}
\titleformat{\subsubsection}
{\sffamily\normalsize\bfseries}{\thesubsubsection}{1em}{}

%Nastavení zvýrazňování kódu v \lslisting
\definecolor{mygreen}{rgb}{0,0.6,0}
\definecolor{mygray}{rgb}{0.5,0.5,0.5}
\lstset{commentstyle=\color{mygreen},keywordstyle=\color{blue},numberstyle=\tiny\color{mygray}}

\author{Štěpán Ševčík}

\begin{document}

%-------------Úvodni strana---------------
\begin{titlepage}

\includegraphics[width=50mm]{img/FAV.jpg}
\\[160 pt]
\centerline{ \Huge \sc KIV/PSI - Počítačové Sítě}
\centerline{ \huge \sc Semestrální práce}
\\[12 pt]
{\large \sc
\centerline{Klient SNMP}
}


{
\vfill 
\parindent=0cm
\textbf{Jméno:} Štěpán Ševčík\\
\textbf{Osobní číslo:} A17B0087P\\
\textbf{E-mail:} kiwi@students.zcu.cz\\
\textbf{Datum:} {\large \today\par} %datum

}

\end{titlepage}

%------------------Obsah-------------------
\newpage
\setcounter{page}{2}
\setcounter{tocdepth}{3}
\tableofcontents
%------------------------------------------

%--------------Text dokumentu--------------


\section{Zadání}
Sestavte program pro čtení a zobrazení obsahu jedné nebo více tabulek MIB.
Čtěte je ty sloupce, které budete zobrazovat, tzn. že pro vybrané tabulky nebudete zobrazovat všechny sloupce, ale pouze např. 4 až 5 sloupců tak, aby zobrazovaná informace dávala smysl.

\subsection{SNMP}
SNMP (Simple Network Management Protocol) je protokol pro správu zařízení po síti. Pomocí tohoto protokolu lze číst a hlídat hodnoty zařízení adresovaných pomocí objektových identifikátorů (OID).
Struktura objektů je zřetězena do hiearchického stromu, kde například \texttt{1.3.6.1.2.1.4.21.1} je identifikátor tabulky síťových rozhraní.
Tabulky obsahují sloupce, které opět respektují hiearchický model, a tedy například sloupec cílové adresy je v \texttt{1.3.6.1.2.1.4.21.1.1}.

K hodnotám lze přistupovat pomocí zpráv GET resp. GETNEXT.

\section{Návrh}
\subsection{Programové rozhraní}
Úlohou je čtení obsahu SNMP tabulky, ježjíž identifikátorem je OID. Dalším z požadavků je volitelnost sloupců.

Prvotní nápad je vytvoření struktury Tabula, která bude obsahovat vlastní OID a množinu sloupců, což je struktura obsahující název a konkretizující část OID.

Jelikož není třeba s daty provádět komplexnější operace, stačí získat a rovnou vypsat.

\subsection{Práce se SNMP}
Pro získávání dat je potřeba mít SNMP klienta, který bude řešit převod aplikačních požadavků na data protokolu.
Přenašečem dat protokolu je PDU (tedy Protocol Data Unit), který lze zjednodušeně chápat jako kolekci OID a hodnot.

SNMP požadavek obsahuje PDU obsahující výčet požadovaných OID, zatímco odpověď obsahuje množinu získaných OID a jejich hodnot.
V určitých případech však tyto množiny mohou být rozlišné, například v případě volání GET a neexistence požadovaného OID nebo v případě volání GETNEXT.

Požadavkem je průchod tabulkou, konkrétní OID k zjištění nejsou známy a proto je třeba volat GETNEXT s OID požadovaných sloupců.
Při získávání následující hodnoty je třeba opět volat GETNEXT, již s OID, které bylo získáno v minulém volání.

Nyní je však třeba dbát pozor na získanou hodnotu protože například při procházení tabulky \text{1.3.6.1.2.4}, sloupce \texttt{2}, získat OID sloupce \texttt{3}.
Znamená to, že v předcházejícím požadavku byla tabulka již prolezena a je to znamení, že má být výpis zastaven.

\section{Implementace}
Pro vypracování této úlohy byl zvolen programovací jazyk Java, pro který již existuje knihovna pro práci se SNMP (org.snmp4j).
Nad touto knihovnou byl vytvořen klient zastřešující požadovanou funkcionalitu protokolu SNMP.

Pro toto řešení byla vytvořena rutina \texttt{TableWalk}, která je parametrizována OID požadované tabulky.
V této rutině jsou dále nastaveny požadované sloupce, tj. jejich OID, název a šířka pro výpis.
Tato rutina je dále spuštěna se SNMP klientem, nad kterým má běžet, a výstupním streamem, do kterého má vypisovat svůj běhový výstup, tj. hlavičku a tělo tabulky.

Použitý SNMP klient zprostředkovává volání do knihovny a ošetřuje výjímečné stavy. Dále klient detailně loguje události pomocí knihovny Log4J, což zahrnuje například volání metod nebo doba trvání požadavků.

Jednoduchá integrace a spuštění obalové knihovny je implementováno třídou \texttt{SnmpDemoMain}, ve kterém je vytvořena rutina procházení routovací a vybrány sloupce čísla rozhraní, typu routy a adresy a masky sítě.

\subsection{Použití}

\noindent Program je k nalezení v adresáři dist a spouští se příkazem:

\texttt{java -jar snmp-demo.jar host (community)}.

\noindent Při nesprávném použití (bez zadání argumentů) program vypisuje nápovědu.
Součástí běhu je vytvoření logovacího souboru, ve kterém jsou uchovány různé běhové a debugovací informace.

\begin{figure}
\centering
\includegraphics[width=10cm]{img/01_cli}
\caption{Ukázkový výstup}
\label{fig:cli}
\end{figure}

\section{Závěr}
Výsledkem této semestrální práce je všeobecně použitelný procházeč SNMP tabulek a podklady pro jednoduché vytvoření nástroje pro práci se SNMP.


%------------------------------------------

\end{document}
